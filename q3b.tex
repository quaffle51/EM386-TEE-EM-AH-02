% !TeX root = ./TMA02.tex

\begin{table}[!htp]\centering
\begin{tabular}{lrr|cr}\toprule
$v_1$ &$v_2$ &$v_3$ &$f(v_.v_2.v_3)$ \\\midrule
0 &0 &0 &0 \\
1 &0 &0 &1 \\
0 &1 &0 &1 \\
1 &1 &0 &0 \\
0 &0 &1 &1 \\
1 &0 &1 &0 \\
0 &1 &1 &1 \\
1 &1 &1 &1 \\
\bottomrule
\end{tabular}
\caption{Truth table for function $f:V(3,2) \rightarrow V(1,2)$.}\label{tab:4}
\end{table}
We obtain an expression for $f(v_.v_2.v_3)$ as a Boolean multinomial in three Boolean variables as follows.

Making use of: $x\lor y = x + y +xy$; $x + x = 0$; $xx=x$; and $x + x=0$ in $GF(2)$, gives
\begin{align*}
f(v_.v_2.v_3) &= \bar{v_1}\bar{v_2}v_3\; \lor
                 \bar{v_1}v_2\bar{v_3}\; \lor
                 \bar{v_1}v_2v_3\; \lor
                 v_1\bar{v_2}\bar{v_3} \lor
                 v_1 v_2 v_3\\
              &= \bar{v_1}v_3(v_2+\bar{v_2}) + 
                 \bar{v_1}v_2(v_3+\bar{v_3}) +
                 \bar{v_2}v_3(v_1+\bar{v_1}) +
                 v_1\bar{v_2}\bar{v_3}\\
              &= \bar{v_1}v_3 + 
                 \bar{v_1}v_2 +
                 \bar{v_2}v_3 +
                 v_1\bar{v_2}\bar{v_3}\\
              &= (1+v_1)v_3 + 
                 (1+v_1)v_2 +
                 (1+v_2)v_3 +
                 v_1(1+v_2)(1+v_3)\\
              &= v_3+v_1v_3 + 
                 v_2+v_1v_2 +
                 v_3 +v_2v_3+
                 (v_1+v_1v_2)(1+v_3)\\
             &= v_3+v_1v_3 + 
                 v_2+v_1v_2 +
                 v_3 +v_2v_3+
                 v_1+v_1v_2 + v_1v_3+v_1v_2v_3\\
             &= v_1 + v_2 + v_3 + v_2v_3 + v_1v_2v_3.                
\end{align*}

