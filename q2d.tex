% !TeX root = ./TMA02.tex
\begin{octavecode}
	q = 7;
	y1 = [1 1 0 5 2 5];
	y2 = [1 0 4 6 6 4];
	A = [4 6 6 4;3 6 3 1];
	a11 = A(1,1);
	a12 = A(1,2);
	a13 = A(1,3);
	a14 = A(1,4);
	
	a21 = A(2,1);
	a22 = A(2,2);
	a23 = A(2,3);
	a24 = A(2,4);
	G = [eye(2),A];
	
	v = [4, 5];
	test1 = mod(v*G, q);
	GT = G';
	check = mod(H*G', q);
\end{octavecode}
The generator matrix in standard form is of the form $G = [I_{t}|A_{t\times n-t}]$ where $t = 2$ and $n=6$, i.e.
\[
	G =
	\begin{pmatrix}[cc|cccc]
	1 & 0 & a_{1,1} & a_{1,2} & a_{1,3} & a_{1,4} \\ 
	0 & 1 & a_{2,1} & a_{2,2} & a_{2,3} & a_{2,4}
	\end{pmatrix} 
\]
We have a valid codeword namely $\bm{y_1} = \octavec{disp(y1)}$ from part~(c). Thus, to generate this codeword we calculate
\begin{align*}
			\bm{y} &= [\octavec{disp(y1(1,1))}\ \octavec{disp(y1(1,2))}\ ]\;
							\begin{pmatrix}[cc|cccc]
	1 & 0 & a_{1,1} & a_{1,2} & a_{1,3} & a_{1,4} \\ 
	0 & 1 & a_{2,1} & a_{2,2} & a_{2,3} & a_{2,4}
	\end{pmatrix},\\
					&=  
					 [\octavec{disp(y1(1,1))}\;
					\octavec{disp(y1(1,2))}\;\;
					a_{1,1} + a_{2,1}\;\;
					a_{1,2} + a_{2,2}\;\;
					a_{1,3} + a_{2,3}\;\;
					a_{1,4} + a_{2,4}
					].
\end{align*}
We also have the valid codeword $\bm{y_2} = 104664$ obtained in part~(b). Thus, to generate this codeword we calculate
\begin{align*}
			\bm{y} &= [1 0]\;
							\begin{pmatrix}[cc|cccc]
	1 & 0 & a_{1,1} & a_{1,2} & a_{1,3} & a_{1,4} \\ 
	0 & 1 & a_{2,1} & a_{2,2} & a_{2,3} & a_{2,4}
	\end{pmatrix},\\
	&=[1\;\;0\;\;a_{1,1}\;\;a_{1,2}\;\;a_{1,3}\;\;a_{1,4}],
\end{align*}
and it immediately follows that $a_{1,1}=\octavec{disp(a11)},\; 
                                 a_{1,2}=\octavec{disp(a12)},\;  
                                 a_{1,3}=\octavec{disp(a13)},\;
                                 a_{1,4}=\octavec{disp(a14)},\;
                                 a_{2,1}=\octavec{disp(a21)},\; 
                                 a_{2,2}=\octavec{disp(a22)},\;  
                                 a_{2,3}=\octavec{disp(a23)},\textrm{ and }
                                 a_{2,4}=\octavec{disp(a24)}$. Thus, the generator matrix in standard form for the code $C$ is
\[
	G = 
	\begin{pmatrix}
	\octavec{disp(G(1,1))} & 
	\octavec{disp(G(1,2))} &  
	\octavec{disp(G(1,3))} &  
	\octavec{disp(G(1,4))} &  
	\octavec{disp(G(1,5))} & 
	\octavec{disp(G(1,6))} \\ 
	\octavec{disp(G(2,1))} & 
	\octavec{disp(G(2,2))} &  
	\octavec{disp(G(2,3))} &  
	\octavec{disp(G(2,4))} &  
	\octavec{disp(G(2,5))} & 
	\octavec{disp(G(2,6))}
	\end{pmatrix}.
\]
\begin{comment}
Check using the vector $[\octavec{disp(v(1,1))}\;\;
	  \octavec{disp(v(1,2))}]$
\[
	[\octavec{disp(v(1,1))}\;\;
	  \octavec{disp(v(1,2))}]
	  \begin{pmatrix}
	\octavec{disp(G(1,1))} & 
	\octavec{disp(G(1,2))} &  
	\octavec{disp(G(1,3))} &  
	\octavec{disp(G(1,4))} &  
	\octavec{disp(G(1,5))} & 
	\octavec{disp(G(1,6))} \\ 
	\octavec{disp(G(2,1))} & 
	\octavec{disp(G(2,2))} &  
	\octavec{disp(G(2,3))} &  
	\octavec{disp(G(2,4))} &  
	\octavec{disp(G(2,5))} & 
	\octavec{disp(G(2,6))}
	\end{pmatrix}
	 = \octavec{disp(test1)}.
\]
Table~\ref{tab:4} shows the result of calculating the syndrome of the vector $\octavec{disp(test1)}$ using the generator matrix, $G$. As the calculated syndrome is zero the vector $\octavec{disp(test1)}$ is a valid codeword.  This gives us confidence that $G$ is correct.

\begin{table}[!htp]\centering
\begin{tabular}{l|rrrrrr|rr}\toprule
$i\rightarrow$ &1 &2 &3 &4 &5 &6 & \\\cmidrule{1-7}
$j\downarrow$ &4 &5 &3 &5 &4 &0 &$S_j$ \\\midrule
1 &4 &5 &3 &5 &4 &0 &0 \\
2 &4 &10 &9 &20 &20 &0 &0 \\
3 &4 &20 &27 &80 &100 &0 &0 \\
4 &4 &40 &81 &320 &500 &0 &0 \\
\bottomrule
\end{tabular}
\caption{Table showing that the vector $\octavec{disp(test1)}$ is a valid codeword as its syndrome is $\bm{0}$.}\label{tab:4}
\end{table}

Check: If $G$ is correct then every row of $H$ is orthogonal to every row of $G$. That is, $HG^{T} = \bm{0}$\marginnote{\hill pp.70-71}

	\[
HG^T=
\begin{pmatrix}
1 & 1 & 1 & 1 & 1 & 1 \\ 
1 & 2 & 3 & 4 & 5 & 6 \\ 
1^2 & 2^2 & 3^2 & 4^2 & 5^2 & 6^2 \\ 
1^3 & 2^3 & 3^3 & 4^3 & 5^3 & 6^3
\end{pmatrix}
\begin{pmatrix}
\octavec{disp(GT(1,1))} & \octavec{disp(GT(1,2))} \\ 
\octavec{disp(GT(2,1))} & \octavec{disp(GT(2,2))} \\
\octavec{disp(GT(3,1))} & \octavec{disp(GT(3,2))} \\
\octavec{disp(GT(4,1))} & \octavec{disp(GT(4,2))} \\
\octavec{disp(GT(5,1))} & \octavec{disp(GT(5,2))} \\
\octavec{disp(GT(6,1))} & \octavec{disp(GT(6,2))}
\end{pmatrix} =
\begin{pmatrix}
\octavec{disp(check(1,1))} & \octavec{disp(check(1,2))} \\ 
\octavec{disp(check(2,1))} & \octavec{disp(check(2,2))} \\ 
\octavec{disp(check(3,1))} & \octavec{disp(check(3,2))} \\ 
\octavec{disp(check(4,1))} & \octavec{disp(check(4,2))} \\ 
\end{pmatrix}.
\]
\end{comment}
