% !TeX root = ./TMA02.tex
In order to show that
\[
H_{ham} =
\begin{pmatrix}
0 & 1 & 2 & 4 & 6 & 4 & 3 & 5 \\ 
3 & 2 & 2 & 6 & 1 & 2 & 2 & 0
\end{pmatrix} 
\]
is a parity check matrix for a Hamming code Ham(2,7) consider the following.

For Ham(2,7) we have $r=2$ and $q=7$ so any non-zero vector $\bm{v}$ in $V(2,7)$ has exactly $7-1=6$ non-zero scalar multiples, forming the set $\{\lambda \bm{v}|\lambda \in GF(7), \lambda\neq 0\}$. The $(7^2-1)/(7-1)=\pyc{print(int((7**2-1)/(7-1)))}$ such sets or classes are given below. 

\[\left\{\begin{pmatrix}
0 \\ 3 
\end{pmatrix},\quad
\begin{pmatrix}
0 \\ 6 
\end{pmatrix},\quad
\begin{pmatrix}
0 \\ 2 
\end{pmatrix},\quad
\begin{pmatrix}
0 \\ 5 
\end{pmatrix},\quad
\begin{pmatrix}
0 \\ 1 
\end{pmatrix},\quad
\begin{pmatrix}
0 \\ 4 
\end{pmatrix}
\right\}
\]
\[\left\{\begin{pmatrix}
1 \\ 2 
\end{pmatrix},\quad
\begin{pmatrix}
2 \\ 4 
\end{pmatrix},\quad
\begin{pmatrix}
3 \\ 6 
\end{pmatrix},\quad
\begin{pmatrix}
4 \\ 1 
\end{pmatrix},\quad
\begin{pmatrix}
5 \\ 3 
\end{pmatrix},\quad
\begin{pmatrix}
6 \\ 5 
\end{pmatrix}
\right\}
\]
\[\left\{\begin{pmatrix}
2 \\ 2 
\end{pmatrix},\quad
\begin{pmatrix}
4 \\ 4 
\end{pmatrix},\quad
\begin{pmatrix}
6 \\ 6 
\end{pmatrix},\quad
\begin{pmatrix}
1 \\ 1 
\end{pmatrix},\quad
\begin{pmatrix}
3 \\ 3 
\end{pmatrix},\quad
\begin{pmatrix}
5 \\ 5 
\end{pmatrix}
\right\}
\]
\[\left\{\begin{pmatrix}
4 \\ 6 
\end{pmatrix},\quad
\begin{pmatrix}
1 \\ 5 
\end{pmatrix},\quad
\begin{pmatrix}
5 \\ 4 
\end{pmatrix},\quad
\begin{pmatrix}
2 \\ 3 
\end{pmatrix},\quad
\begin{pmatrix}
6 \\ 2 
\end{pmatrix},\quad
\begin{pmatrix}
3 \\ 1 
\end{pmatrix}
\right\}
\]
\[\left\{\begin{pmatrix}
6 \\ 1 
\end{pmatrix},\quad
\begin{pmatrix}
5 \\ 2 
\end{pmatrix},\quad
\begin{pmatrix}
4 \\ 3 
\end{pmatrix},\quad
\begin{pmatrix}
3 \\ 4 
\end{pmatrix},\quad
\begin{pmatrix}
2 \\ 5 
\end{pmatrix},\quad
\begin{pmatrix}
1 \\ 6 
\end{pmatrix}
\right\}
\]
\[\left\{\begin{pmatrix}
4 \\ 2 
\end{pmatrix},\quad
\begin{pmatrix}
1 \\ 4 
\end{pmatrix},\quad
\begin{pmatrix}
5 \\ 6 
\end{pmatrix},\quad
\begin{pmatrix}
2 \\ 1 
\end{pmatrix},\quad
\begin{pmatrix}
6 \\ 3 
\end{pmatrix},\quad
\begin{pmatrix}
3 \\ 5 
\end{pmatrix}
\right\}
\]
\[\left\{\begin{pmatrix}
3 \\ 2 
\end{pmatrix},\quad
\begin{pmatrix}
6 \\ 4 
\end{pmatrix},\quad
\begin{pmatrix}
2 \\ 6 
\end{pmatrix},\quad
\begin{pmatrix}
5 \\ 1 
\end{pmatrix},\quad
\begin{pmatrix}
1 \\ 3 
\end{pmatrix},\quad
\begin{pmatrix}
4 \\ 5 
\end{pmatrix}
\right\}
\]
\[\left\{\begin{pmatrix}
5 \\ 0 
\end{pmatrix},\quad
\begin{pmatrix}
3 \\ 0 
\end{pmatrix},\quad
\begin{pmatrix}
1 \\ 0 
\end{pmatrix},\quad
\begin{pmatrix}
6 \\ 0 
\end{pmatrix},\quad
\begin{pmatrix}
4 \\ 0 
\end{pmatrix},\quad
\begin{pmatrix}
2 \\ 0 
\end{pmatrix}
\right\}
\]

Each class contains exactly one column vector from the parity-check matrix $H$ and as such each column vectors of $H$ is linearly independent of any other. Thus, the given parity-check matrix, $H$, is that for a Hamming code Ham(2,7).