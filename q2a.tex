% !TeX root = ./TMA02.tex
Let $C$ be the code over $GF(q)$ defined to have the parity-check matrix
\[
H=
\begin{pmatrix}
1^0 & 1^0 & \ldots & 1^0 \\ 
1^1 & 2^1 & \ldots & n^1 \\ 
1^2 & 2^2 & \ldots & n^2 \\ 
\vdots & \vdots & \ldots & \vdots \\ 
1^{d-2} & 2^{d-2} & \ldots & n^{d-2}
\end{pmatrix},
\]
where $d \leq n \leq q-1$. Any $d-1$ columns of $H$ form a Vandermonde matrix and so are linearly independent by \textbf{Theorems~11.1} and \textbf{11.2}\marginnote{\hill pages 126-7.}. Hence, by \textbf{Theorem 8.4}\marginnote{\hill page 85.}, $C$ has a minimum distance $d$ and so is a $q$-ary $(n, q^{n-d+1, d})$-code.

We are given 
\[
H=
\begin{pmatrix}
1 & 1 & 1 & 1 & 1 & 1 \\ 
1 & 2 & 3 & 4 & 5 & 6 \\ 
1^2 & 2^2 & 3^2 & 4^2 & 5^2 & 6^2 \\ 
1^3 & 2^3 & 3^3 & 4^3 & 5^3 & 6^3
\end{pmatrix}.
\]
and as such it is seen that $n=6$, $3 = d - 2$ and hence $d=5$. The code is over $GF(7)$ so that $q=7$. Thus, the code has a minimum distance of $5$ and is a $7$-ary $(6, 7^{6-5+1},5)$-code, that is $(6, 7^2,5)$-code over $GF(7)$ and as such the dimension of the code $k=2$. Also, $d = 2t + 1$ and therefore $t=(5 - 1)/2 =\octavec{disp((5-1)/2)}$. So we have a  $\octavec{disp((5-1)/2)}$-error-correcting code of length $6$ over $GF(7)$. 