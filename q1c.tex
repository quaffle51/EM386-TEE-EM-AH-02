% !TeX root = ./TMA02.tex
Sim(2,7) has $\left|V(8,7)\right|/\left|\textrm{Sim}(2,7)\right| = 7^8/7^2 = 7^6$ cosets.\marginnote{\textbf{Theorem~6.4} \hill~p57.}[-0.2cm]

Now in our case, $d(C) = 2t + 1 = 7$,\marginnote{\hill~p74.}[0cm] so we are guaranteed that $<=t, <= 3$ errors can be corrected in any codeword. In the top part of the Slepian standard array the coset-leaders will be those that have a weight of three or less and in this part of the array we will have one vector of weight zero, namely $00000000$. For the case where the weight is one we can choose for a given coordinate position in a vector of length eight any one of the values from $\{1,2,\ldots,6\}$. As there are eight coordinate positions in the vector we have
\[
	8\times\binom{8}{1} = 8\times\frac{8!}{(8-1)!\times1!} = 6\cdot8=48,
\]
coset-leaders of weight one. Continuing with this logic we can build an expression for the number of coset-leaders of weight two and weight three.

Thus, the number of coset-leaders in the top part of the Slepian standard array is given by the following expression:
\[
	\sum_{k=0}^t(q-1)^k\binom{n}{k}=\binom{n}{0} + (q-1)\binom{n}{1} + (q-1)^2\binom{n}{2} +,\ldots,+ (q-1)^t\binom{n}{t}. 
\]
In our case $q=7$, $n=8$ and $t=3$ as $d(C) = 2t + 1 = 7$ and so we have
\begin{align*}
	\sum_{k=0}^3(7-1)^k\binom{8}{k}&=\binom{8}{0} + (7-1)\binom{8}{1} + (7-1)^2\binom{8}{2} + (7-1)^3\binom{8}{3},\\
	&= 1 + \pyc{print(6*8)} + \pyc{print(6**2 * 28)} + \pyc{print(6**3 * 56)} = \pyc{print(1 + 6*8 + 6**2 * 28 + 6**3 * 56)}\quad\textrm{coset-leaders}.
\end{align*}

Thus, there are \pyc{print(1 + 6*8 + 6**2 * 28 + 6**3 * 56)} coset leaders in the top part of the Slepian standard array for Sim(2,7). 


\begin{octavecode}
function H = to_standard_form
  n = 8;
  k = 2;
  G=[0, 1, 2, 4, 6, 4, 3, 5; 3, 2, 2, 6, 1, 2, 2, 0];
  G = [mod(3*G(1,:),7); mod(5*G(2,:),7)];
  G = [G(1,:); mod(G(2,:)-G(1,:),7)];
  G = [mod(5*G(1,:),7);mod(1*G(2,:),7)];
  G = [mod(1*G(2,:),7);mod(1*G(1,:),7)];
  A = G(1:2,3:8);
  H = [mod(-A',7),eye(n-k)];
endfunction;

\end{octavecode}
\begin{octavecode}
function result = syndrome(y, H)
	result = mod(y * H', 7);
endfunction

H = to_standard_form;

y1 = [1, 0, 0, 0, 0, 0, 0, 0];
y2 = [0, 1, 0, 0, 0, 0, 0, 0];
y3 = [0, 0, 1, 0, 0, 0, 0, 0];
y4 = [0, 0, 0, 1, 0, 0, 0, 0];
y5 = [0, 0, 0, 0, 1, 0, 0, 0];
y6 = [0, 0, 0, 0, 0, 1, 0, 0];
y7 = [0, 0, 0, 0, 0, 0, 1, 0];
y8 = [0, 0, 0, 0, 0, 0, 0, 1];

s1 = y1*H';
s2 = y2*H';
s3 = y3*H';
s4 = y4*H';
s5 = y5*H';
s6 = y6*H';
s7 = y7*H';
s8 = y8*H';

y = [4, 5, 6, 3, 2, 0, 3, 6];
s = syndrome(y, H);
\end{octavecode}
Shown in Table~\ref{tab:1 } are the syndromes for the vectors 
\[\octave{disp(y1)},\octave{disp(y2)},\ldots,\octave{disp(y8)}.\] 
To correct the received vector $\octave{disp(y)}$, assuming at most one error, which has syndrome 
$\octavec{disp(s)}$, observe that the syndrome of $\octavec{disp(y2)}\times 3$ is equal to $ \octavec{disp(mod(y2*3*H',7))}\Mod{7}$. Hence, the received vector $\octavec{disp(y)}$ is in the same coset as that with coset-leader $\octavec{disp(y2)} \times 3 =\octavec{disp(y2*3)}$. So, we decode the received vector as $\octavec{disp(y)} - \octavec{disp(y2*3)} = \octavec{disp(y-y2*3)}$.
\begin{table}[!htp]\centering
\begin{tabular}{ccc}\toprule
vector & syndrome \\\midrule
\octavec{disp(y1)} & \octavec{disp(s1)} \\
\octavec{disp(y2)} & \octavec{disp(s2)} \\
\octavec{disp(y3)} & \octavec{disp(s3)} \\
\octavec{disp(y4)} & \octavec{disp(s4)} \\
\octavec{disp(y5)} & \octavec{disp(s5)} \\
\octavec{disp(y6)} & \octavec{disp(s6)} \\
\octavec{disp(y7)} & \octavec{disp(s7)} \\
\octavec{disp(y8)} & \octavec{disp(s8)} \\
\bottomrule
\end{tabular}
\caption{Syndromes of vectors $10000000, 01000000,\ldots,00000001$.}\label{tab:1 }
\end{table}











