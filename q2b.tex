% !TeX root = ./TMA02.tex
\begin{octavecode}
	function i = Inverse(x)
	  T = [
	      0 0 0 0 0 0 0;
	      0 1 2 3 4 5 6;
	      0 2 4 6 1 3 5;
	      0 3 6 2 5 1 4;
	      0 4 1 5 2 6 3;
	      0 5 3 1 6 4 2;
	      0 6 5 4 3 2 1
	    ];
	  i = 0;	
	  if x != 0
	    for j = [1 2 3 4 5 6 7]
	      if T(x+1,j) != 1
	        i++;
	      else
	        break;
	      end
	    end 
	  end
	  
	endfunction
	
	y = [3 2 4 6 6 4];
	
	
	H = [1^0 2^0 3^0 4^0 5^0 6^0;
	     1^1 2^1 3^1 4^1 5^1 6^1;
	     1^2 2^2 3^2 4^2 5^2 6^2;
	     1^3 2^3 3^3 4^3 5^3 6^3
	    ];
	
	H  = mod(H, 7);
	
	y1 = y(1);
	y2 = y(2);
	y3 = y(3);
	y4 = y(4);
	y5 = y(5);
	y6 = y(6);
	
	S = mod(y*H', 7);
	
	S1a = S(1);
	S2a = S(2);
	S3a = S(3);
	S4a = S(4);
	H11 = H(1,1);
	
	function result =  Sj(j, y)
		result = 0;
		for i=[1 2 3 4 5 6]
			result += y(i) * i^(j-1);
		end
		result = mod(result, 7);
	endfunction
	
	S1 = Sj(1,y);
	S2 = Sj(2,y);
	S3 = Sj(3,y);
	S4 = Sj(4,y);
	
	A1 = 4;
	A2 = 1;
	B1 = 4;
	B2 = 2;
	
	#factor (1 + 4x + 2x^2) mod 7
	Z2 = mod(-3,7);
	Z1 = mod(-6,7);
	z1 = Inverse(Z1);
	z2 = Inverse(Z2);
	
	m1 = 2;
	m2 = 2;
	X1 = 1;
	X2 = 2;
	
	v = [m1 m2 4 6 6 4];
	sy = mod(v * H', 7);
\end{octavecode}
The received vector is $\octavec{disp(y)}$ and assume that two errors have occurred.
Suppose that errors have occurred in positions $X_1$ and $X_2$ with respective magnitudes $m_1$ and $m_2$.  If no errors have occurred then $m_1 = m_2 = 0$ and if only one error has occurred then $m_2=0$.

From the received vector $\bm{y} = \octavec{disp(y)}$ calculate the syndrome
\[
	(S_1,S_2,S_3,S_4) = \bm{y}H^T.
\]
That is we calculate\marginnote{\hill page 132.}[1.4cm]
\begin{equation}
	\label{eq:2.1}
	S_j = \sum_{i=1}^6 y_i i^{j-1} = \sum_{i=1}^2 m_i X_i^{j-1} \textrm{ for } j=1,2,3,4,
\end{equation}

which gives the following syndrome for received vector $\bm{y}$ (see Table~\ref{tab:2})
\begin{table}[!htp]\centering
\begin{tabular}{l|rrrrrr|rr}\toprule
$i\rightarrow$ &1 &2 &3 &4 &5 &6 & \\\cmidrule{1-7}
$j\downarrow$ &3 &2 &4 &6 &6 &4 & $S_j$\\\midrule
1 &3 &2 &4 &6 &6 &4 &4 \\
2 &3 &4 &12 &24 &30 &24 &6 \\
3 &3 &8 &36 &96 &150 &144 &3 \\
4 &3 &16 &108 &384 &750 &864 &4 \\
\bottomrule
\end{tabular}
\caption{Calculation of the syndrome from the received vector $\bm{y}$.}\label{tab:2}
\end{table}
\begin{align*}
(S_1,S_2,S_3,S_4) &=  \left( \octavec{disp(S1)}, \octavec{disp(S2)}, \octavec{disp(S3)},\octavec{disp(S4)} \right).
\end{align*}
To find the errors the following systems of equations must be solved for $X_i$ and $m_i$
\begin{align*}
	m_1 + m_2 &= S_1\\
	m_1X_1 + m_2X_2 &= S_2\\
	m_1X_1^2 + m_2X_2^2 &= S_3\\
	m_1X_1^3 + m_2X_2^3 &= S_4
\end{align*}
Assuming at most 2 errors in positions $X_1, X_2$ of respective magnitudes $m_1, m_2$ we have
\[
	\phi(\theta)=  \frac{m_1}{1-X_1\theta}+ \frac{m_2}{1-X_2\theta} = \frac{A_1 + A_2\theta}{1 + B_1\theta + B_2\theta^2}
\]
where, by 11.6 and 11.7 (\hill page~133), the $A_i$ and $B_i$ satisfy
\begin{align*}
	A_1 &= \octavec{disp(S1)}\\
	A_2 &= \octavec{disp(S2)} + \octavec{disp(S1)} B_1\\
	0   &= \octavec{disp(S3)} + \octavec{disp(S2)} B_1 + \octavec{disp(S1)}B_2\\
	0   &= \octavec{disp(S4)} + \octavec{disp(S3)} B_1 + \octavec{disp(S2)} B_2.
\end{align*}
Solving for $B_1$ and $B_2$ first
\begin{align*}
	-3 &\equiv 6B_1 + 4B_2 \Mod{7}\\
	-4 &\equiv 3B_1 + 6B_2 \Mod{7}\\\\
	-3 &\equiv 6B_1 + 4B_2 \Mod{7}\\	
	-8 &\equiv 6B_1 +12B_2 \Mod{7}\\\\
	 4 &\equiv 6B_1 + 4B_2 \Mod{7}\\
	 6 &\equiv 6B_1 + 5B_2 \Mod{7}\\
	 2 &\equiv B_2 \Mod{7}\\\\
	 4 &\equiv 6B_1 + 4\cdot 2 \Mod{7}\\
	 4 &\equiv 6B_1 + 1 \Mod{7}\\
	 3 &\equiv 6B_1 \Mod{7}\\
	 3\cdot6^{-1} &\equiv B_1 \Mod{7}\\
	 3\cdot6 &\equiv B_1 \Mod{7}\\
	 4 &\equiv B_1 \Mod{7}.
\end{align*}
Then for $A_2$
\begin{align*}
	A_2 &\equiv 6 + 4\cdot B_1 \Mod{7}\\
	A_2 &\equiv 6 + 4\cdot \octavec{disp(B1)} \Mod{7}\\
	A_2 &\equiv 6 + \octavec{disp(4 * B1)} \Mod{7}\\
	A_2 &\equiv \octavec{disp(6 + 16)} \Mod{7}\\
	A_2 &\equiv \octavec{disp(mod(6 + 16,7))} \Mod{7}\\
\end{align*}
Thus, $A_1=\octavec{disp(A1)}$, $A_2=\octavec{disp(A2)}$, $B_1=\octavec{disp(B1)}$, and $B_2=\octavec{disp(B2)}$. Therefore,
\begin{align*}
	\phi(\theta) &= \frac{A_1 + A_2\theta}{1 + B_1\theta + B_2\theta^2}\\
	 &= \frac{ \octavec{disp(A1)} + \theta}{1 + \octavec{disp(B1)}\theta + \octavec{disp(B2)}\theta^2}\\
	&= \frac{4 + \theta}{2(\theta+3)(\theta + 6)}\Mod{7}.
\end{align*}
The zeros of the quadratic are $\octavec{disp(Z1)}$ and $\octavec{disp(Z2)}$. The error positions are the inverse of these values, i.e. $X_1=\octavec{disp(z1)}$ and $X_2=\octavec{disp(z2)}$.

To find $m_1$:
\begin{align*}
	\frac{4 + \theta}{(1-\theta)(1-\octavec{disp(z2)}\theta)} &= \frac{m_1}{1-\theta} + \frac{m_2}{1-\octavec{disp(z2)}\theta}\\
	\frac{4 + \theta}{1-2\theta} &= m_1 + \frac{m_2(1-\theta)}{1-2\theta}\\
	\textrm{Let  $\theta = 1$ then},\quad\quad&\\
	\frac{4 + 1}{1 - 2} &\equiv \frac{5}{6} \equiv 5\cdot 6 \equiv 2 \equiv m_1 \Mod{7}\\
	m_1 &\equiv 2\Mod{7}.
\end{align*}
To find $m_2$:
\begin{align*}
	\frac{4 + \theta}{(1-\theta)(1-\octavec{disp(z2)}\theta)} &= \frac{m_1}{1-\theta} +
	\frac{m_2}{1-\octavec{disp(z2)}\theta}\\
	\frac{4 + \theta}{1-\theta} &=  \frac{m_1(1-2\theta)}{1-\theta} + m_2\\
	\textrm{Let  $\theta = 4$ then},\qquad\qquad&\\
	\frac{4 + 4}{1 - 4} &\equiv \frac{1}{4} \equiv 2 \equiv m_2 \Mod{7}\\
	m_2 &\equiv 2\Mod{7}.
\end{align*}
Thus, $m_1=2$, $m_2=2, X_1=1$ and $X_2=2$. As a check we can recalculate the syndrome of the received vector as follows.
\begin{align*}
	S_1 &= m_1 + m_2 = 2 + 2 = 4,\\
	S_2 &= m_1X_1 + m_2X_2 = 2\cdot1+ 2\cdot2 = 6,\\
	S_3 &= m_1X_1^2 + m_2X_2^2 = 2\cdot1^2+ 2\cdot2^2 = 10 \equiv 3 \Mod{7},\\
	S_3 &= m_1X_1^3 + m_2X_2^3 = 2\cdot1^3+ 2\cdot2^3 = 18 \equiv 4 \Mod{7}.\\
\end{align*}
These elements of the syndrome are the same as those previously calculated above using \eqref{eq:2.1}. 

Now to obtain the transmitted codeword from the received vector $\bm{y}$ the error $m_i$ is such that $y_{X_{i}}=x_{X_{i}} + m_i$ which enables us to determine transmitted codeword $\bm{x}$ from the received vector $\bm{y}$.\marginnote{See p.14, $\bm{(6.11)}$ of Block~2 Course Notes.}[-1cm] Thus,
\begin{align*}
	y_{X_{i}}&=x_{X_{i}} + m_i \text{ where } i= 1,2.\\
	y_{X_{1}}&=x_{X_{1}} + m_1,\\
	y_1 &= x_1 + m_1,\\
	3 &= x_1 + 2,\\
	x_1 &= 1.
\end{align*}
\begin{align*}
	y_{X_{2}}&=x_{X_{2}} + m_2,\\
	y_2 &= x_2 + m_2,\\
	2 &= x_2 + 2,\\
	x_2 &= 0.
\end{align*}
Hence, the transmitted codeword was $104664$. This can be checked by calculating its syndrome, which if it is a valid codeword, will the $\bm{0}$. The calculation is summarised in Table~\ref{tab:3} where it is seen that the syndrome is indeed $\bm{0}$.
\begin{table}[!htp]\centering
\begin{tabular}{l|rrrrrr|rr}\toprule
$i\rightarrow$ &1 &2 &3 &4 &5 &6 & \\\cmidrule{1-7}
$j\downarrow$ &1 &0 &4 &6 &6 &4 &$S_j$ \\\midrule
1 &1 &0 &4 &6 &6 &4 &0 \\
2 &1 &0 &12 &24 &30 &24 &0 \\
3 &1 &0 &36 &96 &150 &144 &0 \\
4 &1 &0 &108 &384 &750 &864 &0 \\
\bottomrule
\end{tabular}
\caption{Calculation of the syndrome from the transmitted vector $\bm{y}=104664$.}\label{tab:3}
\end{table}


