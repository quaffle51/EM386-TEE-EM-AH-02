% !TeX root = ./TMA02.tex
The generator matrix for $RM(1, 4)$ is given as  follows
\marginnote{See Example~7.7 B2CN's p.36.}
\[
G = 
\begin{pmatrix}[lrrrrrrrrrrrrrrrr]
1 &1 &1 &1 &1 &1 &1 &1 &1 &1 &1 &1 &1 &1 &1 &1 \\
0 &1 &0 &1 &0 &1 &0 &1 &0 &1 &0 &1 &0 &1 &0 &1 \\
0 &0 &1 &1 &0 &0 &1 &1 &0 &0 &1 &1 &0 &0 &1 &1 \\
0 &0 &0 &0 &1 &1 &1 &1 &0 &0 &0 &0 &1 &1 &1 &1 \\
0 &0 &0 &0 &0 &0 &0 &0 &1 &1 &1 &1 &1 &1 &1 &1 \\
\end{pmatrix}.
\]
To obtain the equations needed to apply the Reed decoding algorithm we need to find first the values $x_1, x_2, x_3, x_4$ and $x_5$ as follows.
\[
	\bm{x}=
	\begin{pmatrix}[lrrrrr]
	a_0 & a_1 & a_2 & a_3 & a_4 
	\end{pmatrix}
	\begin{pmatrix}[lrrrrrrrrrrrrrrrr]
     1 &1 &1 &1 &1 &1 &1 &1 &1 &1 &1 &1 &1 &1 &1 &1 \\
     0 &1 &0 &1 &0 &1 &0 &1 &0 &1 &0 &1 &0 &1 &0 &1 \\
     0 &0 &1 &1 &0 &0 &1 &1 &0 &0 &1 &1 &0 &0 &1 &1 \\
     0 &0 &0 &0 &1 &1 &1 &1 &0 &0 &0 &0 &1 &1 &1 &1 \\
     0 &0 &0 &0 &0 &0 &0 &0 &1 &1 &1 &1 &1 &1 &1 &1 \\
    \end{pmatrix}.
\]
Thus,
\begin{align}
\label{eq:3.1}
	x_0 &= a_0\\
	\label{eq:3.2}
	x_1 &= a_0 + a_1\\
	\label{eq:3.3}
	x_2 &= a_0 + a_2\\
	\label{eq:3.4}
	x_3 &= a_0 + a_1 + a_2\\
	\label{eq:3.5}
	x_4 &= a_0 + a_3\\
	\label{eq:3.6}
	x_5 &= a_0 + a_1 + a_3\\
	\label{eq:3.7}
	x_6 &= a_0 + a_2 + a_3\\
	\label{eq:3.8}
	x_7 &= a_0 + a_1 + a_2 + a_3\\
	\label{eq:3.9}
	x_8 &= a_0 + a_4\\
	\label{eq:3.10}
	x_9 &= a_0 + a_1 + a_4\\
	\label{eq:3.11}
	x_{10} &= a_0 + a_2 + a_4\\
	\label{eq:3.12}
	x_{11} &= a_0 + a_1 + a_2 + a_4\\
	\label{eq:3.13}
	x_{12} &= a_0 + a_3 + a_4\\
	\label{eq:3.14}
	x_{13} &= a_0 + a_1 + a_3 +a_4\\
	\label{eq:3.14}
	x_{14} &= a_0 + a_2 +a_3 + a_4\\
	\label{eq:3.16}
	x_{15} &= a_0 + a_1 + a_2 + a_3 + a_4
\end{align}
\begin{enumerate}[label=(\roman*)]
\item The equations needed to apply the Reed decoding algorithm are as follows.
\begin{align}
	a_1 &= x_0 + x_1\\
	a_1 &= x_2 + x_3\\
	a_1 &= x_4 + x_5\\
	a_1 &= x_6 + x_7\\
	a_1 &= x_8 + x_9\\
	a_1 &= x_{10} + x_{11}\\
	a_1 &= x_{12} + x_{13}\\
	a_1 &= x_{14} + x_{15}
\end{align}
\begin{align}
	a_2 &= x_0 + x_2\\
	a_2 &= x_1 + x_3\\
	a_2 &= x_4 + x_6\\
	a_2 &= x_5 + x_7\\
	a_2 &= x_8 + x_{10}\\
	a_2 &= x_{9} + x_{11}\\
	a_2 &= x_{12} + x_{14}\\
	a_2 &= x_{13} + x_{15}
\end{align}
\begin{align}
	a_3 &= x_0 + x_4\\
	a_3 &= x_1 + x_5\\
	a_3 &= x_2 + x_6\\
	a_3 &= x_3 + x_7\\
	a_3 &= x_8 + x_{12}\\
	a_3 &= x_{9} + x_{13}\\
	a_3 &= x_{10} + x_{14}\\
	a_3 &= x_{11} + x_{15}
\end{align}
\begin{align}
	a_4 &= x_0 + x_8\\
	a_4 &= x_1 + x_9\\
	a_4 &= x_2 + x_{10}\\
	a_4 &= x_3 + x_{11}\\
	a_4 &= x_4 + x_{12}\\
	a_4 &= x_{5} + x_{13}\\
	a_4 &= x_{6} + x_{14}\\
	a_4 &= x_{7} + x_{15}
\end{align}
The first equation for $a_0$ is simply $a_0 = x_0$. To obtain the remaining fifteen equations for $a_0$ we make use of the sixteen equations $x_0$ to $x_{15}$ \eqref{eq:3.1} to \eqref{eq:3.16}, respectively. Explaining how the equations for $a_0$ are determined is best illustrated by an example of how we find one of them. Consider the expression for $x_7$ given by \eqref{eq:3.7}:

\begin{equation}
	a_7 = a_0 + a_1 + a_2 + a_3.
\end{equation}
Now, we have eight expressions each for $a_1$, $a_2$, $a_3$ and $a_4$, and the strategy we take is to maximise the number of terms in the expression for $a_0$, thus we choose $a_1 = x_0 + x_1$, $a_2 = x_4 + x_6$, and $x_3 = x_8 + x_{12}$, to give
\begin{equation}
\label{eq:3.50}
	x_7 = a_0 + x_0 + x_1 + x_4 + x_6 + x_8 + x_{12}.
\end{equation}
Then, rearranging \eqref{eq:3.50} in terms of $a_0$ we obtain
\begin{equation}
	a_0 = x_7 - (x_0 + x_1 + x_4 + x_6 + x_8 + x_{12}).
\end{equation}
As each  $x_i \in GF(2)$ and we are using modulo 2 arithmetic we obtain
\begin{equation}
	a_0 = x_0 + x_1 + x_4 + x_6 + x_7 + x_8 + x_{12}.
\end{equation}
However, do note that it is possible to generate duplicate equations. In such cases a different choice for one or more of $a_1, a_2,\ldots,a_4$ should be made, while still trying to maximise the number of terms in the expression for $a_0$, for substitution into the expression for $x_i$.

The remaining equations for $a_0$ are found in a similar fashion.
\item
To determine the original message word from the received vector $1100 1111 0011 1011$ assuming at most three transmission errors, we first determine the majority votes for each of $a_1$, $a_2$, $a_3$ and $a_4$. 
\begin{octavecode}
	x = [1 1 0 0 1 1 1 1 0 0 1 1 1 0 1 1];
	x_0 = x(1,1);
	x_1 = x(1,2);
	x_2 = x(1,3);
	x_3 = x(1,4);
	x_4 = x(1,5);
	x_5 = x(1,6);
	x_6 = x(1,7);
	x_7 = x(1,8);
	x_8 = x(1,9);
	x_9 = x(1,10);
	x_10 = x(1,11);
	x_11 = x(1,12);
	x_12 = x(1,13);
	x_13 = x(1,14);
	x_14 = x(1,15);
	x_15 = x(1,16);
	
	a = [0 0 1 0 1; 1 0 1 0 1];
	a0_1=a(1,1);
	a1_1=a(1,2);
	a2_1=a(1,3);
	a3_1=a(1,4);
	a4_1=a(1,5);
	
	a0_2=a(2,1);
	a1_2=a(2,2);
	a2_2=a(2,3);
	a3_2=a(2,4);
	a4_2=a(2,5);
	a_1 = sprintf("\%d & \%d & \%d & \%d & \%d", a0_1,a1_1,a2_1,a3_1,a4_1);
	a_2 = sprintf("\%d & \%d & \%d & \%d & \%d", a0_2,a1_2,a2_2,a3_2,a4_2);
	G= [
         1  1  1  1  1  1  1  1  1  1  1  1  1  1  1  1  ;
         0  1  0  1  0  1  0  1  0  1  0  1  0  1  0  1  ;
         0  0  1  1  0  0  1  1  0  0  1  1  0  0  1  1  ;
         0  0  0  0  1  1  1  1  0  0  0  0  1  1  1  1  ;
         0  0  0  0  0  0  0  0  1  1  1  1  1  1  1  1 
       ];
    v1 = a(1,:);
    xx_1 = mod(v1*G,2);
    
    v2 = a(2,:);
    xx_2 = mod(v2*G,2);
\end{octavecode}
\begin{comment}
\begin{align*}
	a_0 &= x_0 = \octavec{disp(x_0)}\\
	a_0 &= x_1 + x_{14} + x_{15}\\
	&= \octavec{disp(x_1)} + 
	   \octavec{disp(x_14)} + 
	   \octavec{disp(x_15)} = \octavec{disp(mod(x_1 + x_14+x_15,2))}\\
	a_0 &= x_2 + x_{13} + x_{15}\\
	&= \octavec{disp(x_2)} + 
	   \octavec{disp(x_13)} + 
	   \octavec{disp(x_15)} = \octavec{disp(mod(x_2 + x_13+x_15,2))}\\
	a_0 &= x_3 + x_9 + x_{11} + x_{14} + x_{15}\\
	&= \octavec{disp(x_3)} + 
	   \octavec{disp(x_9)} + 
	   \octavec{disp(x_11)}+
	   \octavec{disp(x_14)}+ 
	   \octavec{disp(x_15)} = \octavec{disp(mod(x_3 + x_9+x_11+x_14+x_15,2))}\\
	a_0 &= x_4 + x_{11} + x_{15}\\
	&= \octavec{disp(x_4)} + 
	   \octavec{disp(x_11)} + 
	   \octavec{disp(x_15)} = \octavec{disp(mod(x_4 + x_11+x_15,2))}\\
	a_0 &= x_5 + x_9 + x_{13} + x_{14} + x_{15}\\
	&= \octavec{disp(x_5)} + 
	   \octavec{disp(x_9)} + 
	   \octavec{disp(x_13)} + 
	   \octavec{disp(x_14)} + 
	   \octavec{disp(x_15)} = \octavec{disp(mod(x_5 + x_9 + x_13 + x_14 + x_15,2))}\\
	a_0 &= x_6 + x_{10} + x_{13} + x_{14} + x_{15}\\
	&= \octavec{disp(x_6)} + 
	   \octavec{disp(x_10)} + 
	   \octavec{disp(x_13)} + 
	   \octavec{disp(x_14)} + 
	   \octavec{disp(x_15)} = \octavec{disp(mod(x_6 + x_10 + x_13 + x_14 + x_15,2))}\\
	a_0 &= x_7 + x_8 + x_9 + x_{11} + x_{12} + x_{14} + x_{15}\\
	&= \octavec{disp(x_7)} + 
	  \octavec{disp(x_8)} + 
	  \octavec{disp(x_9)} + 
	  \octavec{disp(x_11)} + 
	  \octavec{disp(x_12)} + 
	  \octavec{disp(x_14)} + 
	  \octavec{disp(x_15)} = \octavec{disp(mod(x_7 + x_8 + x_9 + x_11 + x_12 + x_14+x_15,2))}\\
	a_0 &= x_7 + x_8 + x_{15}
	= \octavec{disp(x_7)} + \octavec{disp(x_8)} + \octavec{disp(x_15)} = \octavec{disp(mod(x_7 + x_8+x_15,2))}\\
	a_0 &= x_5 + x_9 + x_{13} + x_{14} + x_{15}\\
	&= \octavec{disp(x_5)} + \octavec{disp(x_9)} + \octavec{disp(x_13)} + \octavec{disp(x_14)} + \octavec{disp(x_15)} = \octavec{disp(mod(x_5 + x_9+x_13+x_14+x_15,2))}\\
	a_0 &= x_6 + x_{10} + x_{13} + x_{14} + x_{15}\\
	&= \octavec{disp(x_6)} + 
	  \octavec{disp(x_10)} + 
	  \octavec{disp(x_13)} + 
	  \octavec{disp(x_14)} + 
	  \octavec{disp(x_15)} = 
	  \octavec{disp(mod(x_6 + x_10 + x_13 + x_14 + x_15,2))}\\
	a_0 &= x_5 + x_8 + x_{10} + x_{11} + x_{13} + x_{14} + x_{15},\\
	a_0 &= x_6 + x_{11} + x_{12} + x_{14} + x_{15}\\
	&= \octavec{disp(x_6)} + 
	  \octavec{disp(x_11)} + 
	  \octavec{disp(x_12)} + 
	  \octavec{disp(x_14)} + 
	  \octavec{disp(x_15)}
	   = \octavec{disp(mod(x_6 + x_11 + x_12 + x_14 + x_15,2))}\\
	a_0 &= x_3 + x_8 + x_{11} + x_{12} + x_{13} + x_{14} + x_{15}\\
	&= \octavec{disp(x_3)} + 
	  \octavec{disp(x_8)} + 
	  \octavec{disp(x_11)} + 
	  \octavec{disp(x_12)} + 
	  \octavec{disp(x_13)} +
	  \octavec{disp(x_14)} +
	  \octavec{disp(x_15)} 
	  = 
	  \octavec{disp(mod(x_3 + x_8 + x_11 + x_12 + x_13 + x_14 + x_15,2))}\\
	a_0 &= x_2 + x_8 + x_{10} + x_{12} + x_{13} + x_{14} + x_{15}\\
	&= \octavec{disp(x_2)} + 
	  \octavec{disp(x_8)} + 
	  \octavec{disp(x_10)} + 
	  \octavec{disp(x_12)} + 
	  \octavec{disp(x_13)} +
	  \octavec{disp(x_14)} +
	  \octavec{disp(x_15)} 
	  = 
	  \octavec{disp(mod(x_2 + x_8 + x_10 + x_12 + x_13 + x_14 + x_15,2))}\\
	a_0 &= x_0 + x_8 + x_9 + x_{10} + x_{11} + x_{12} + x_{13} + x_{14} + x_{15}\\
	&= \octavec{disp(x_0)} + 
	  \octavec{disp(x_8)} + 
	  \octavec{disp(x_9)} + 
	  \octavec{disp(x_10)} + 
	  \octavec{disp(x_11)} + 
	  \octavec{disp(x_12)} + 
	  \octavec{disp(x_13)} +
	  \octavec{disp(x_14)} +
	  \octavec{disp(x_15)} 
	  = 
	  \octavec{disp(mod(x_0 + x_8 + +x_9 +x_10 + x_11+ x_12 + x_13 + x_14 + x_15,2))}
\end{align*}
\end{comment}%

Recall from above that the eight equations for $a_1$ were as follows. 
\begin{align*}
	a_1 &= x_0 + x_1\\
	a_1 &= x_2 + x_3\\
	a_1 &= x_4 + x_5\\
	a_1 &= x_6 + x_7\\
	a_1 &= x_8 + x_9\\
	a_1 &= x_{10} + x_{11}\\
	a_1 &= x_{12} + x_{13}\\
	a_1 &= x_{14} + x_{15}
\end{align*}
With, $x_1x_2\ldots x_{15}=1100111100111011$, these give eight values to each $a_1$, namely

\begin{align*}
	a_1 &= x_0 + x_1
	= \octavec{disp(x_0)}+ \octavec{disp(x_1)} = \octavec{disp(mod(x_0+x_1,2))},\\
	a_1 &= x_2 + x_3
	= \octavec{disp(x_2)}+ \octavec{disp(x_3)} = \octavec{disp(mod(x_2+x_3,2))},\\
	a_1 &= x_4 + x_5
	= \octavec{disp(x_4)}+ \octavec{disp(x_5)} = \octavec{disp(mod(x_4+x_5,2))},\\
	a_1 &= x_6 + x_7
	= \octavec{disp(x_6)}+ \octavec{disp(x_7)} = \octavec{disp(mod(x_6+x_7,2))},\\
	a_1 &= x_8 + x_9
	= \octavec{disp(x_8)}+ \octavec{disp(x_9)} = \octavec{disp(mod(x_8+x_9,2))},\\
	a_1 &= x_{10} + x_{11}
	= \octavec{disp(x_10)}+ \octavec{disp(x_11)} = \octavec{disp(mod(x_10+x_11,2))},\\
	a_1 &= x_{12} + x_{13}
	= \octavec{disp(x_12)}+ \octavec{disp(x_13)} = \octavec{disp(mod(x_12+x_13,2))},\\
	a_1 &= x_{14} + x_{15}
	= \octavec{disp(x_14)}+ \octavec{disp(x_15)} = \octavec{disp(mod(x_14+x_15,2))}.
\end{align*}
The majority vote is in favour of $0$ and therefore $a_1 = 0$.  

Repeating the above for $a_2$, $a_3$ and $a_4$ as follows.

With, $x_1x_2\ldots x_{15}=1100111100111011$, these give eight values to each $a_2$, namely
\begin{align*}
	a_2 &= x_0 + x_2
	= \octavec{disp(x_0)}+ \octavec{disp(x_2)} = \octavec{disp(mod(x_0+x_2,2))},\\
	a_2 &= x_1 + x_3
	= \octavec{disp(x_1)}+ \octavec{disp(x_3)} = \octavec{disp(mod(x_1+x_3,2))},\\
	a_2 &= x_4 + x_6
	= \octavec{disp(x_4)}+ \octavec{disp(x_6)} = \octavec{disp(mod(x_4+x_6,2))},\\
	a_2 &= x_5 + x_7
	= \octavec{disp(x_5)}+ \octavec{disp(x_7)} = \octavec{disp(mod(x_5+x_7,2))},\\
	a_2 &= x_8 + x_{10}
	= \octavec{disp(x_8)}+ \octavec{disp(x_10)} = \octavec{disp(mod(x_8+x_10,2))},\\
	a_2 &= x_{9} + x_{11}
	= \octavec{disp(x_9)}+ \octavec{disp(x_11)} = \octavec{disp(mod(x_9+x_11,2))},\\
	a_2 &= x_{12} + x_{14}
	= \octavec{disp(x_12)}+ \octavec{disp(x_14)} = \octavec{disp(mod(x_12+x_14,2))},\\
	a_2 &= x_{13} + x_{15}
	= \octavec{disp(x_13)}+ \octavec{disp(x_15)} = \octavec{disp(mod(x_13+x_15,2))}.
\end{align*}
The majority vote is in favour of $1$ and therefore $a_2 = 1$.

With, $x_1x_2\ldots x_{15}=1100111100111011$, these give eight values to each $a_3$, namely
\begin{align*}
	a_3 &= x_0 + x_4
	= \octavec{disp(x_0)}+ \octavec{disp(x_4)} = \octavec{disp(mod(x_0+x_4,2))},\\
	a_3 &= x_1 + x_5
	= \octavec{disp(x_1)}+ \octavec{disp(x_5)} = \octavec{disp(mod(x_1+x_5,2))},\\
	a_3 &= x_2 + x_6
	= \octavec{disp(x_2)}+ \octavec{disp(x_6)} = \octavec{disp(mod(x_2+x_6,2))},\\
	a_3 &= x_3 + x_7
	= \octavec{disp(x_3)}+ \octavec{disp(x_7)} = \octavec{disp(mod(x_3+x_7,2))},\\
	a_3 &= x_8 + x_{12}
	= \octavec{disp(x_8)}+ \octavec{disp(x_12)} = \octavec{disp(mod(x_8+x_12,2))},\\
	a_3 &= x_{9} + x_{13}
	= \octavec{disp(x_9)}+ \octavec{disp(x_13)} = \octavec{disp(mod(x_9+x_13,2))},\\
	a_3 &= x_{10} + x_{14}
	= \octavec{disp(x_10)}+ \octavec{disp(x_14)} = \octavec{disp(mod(x_10+x_14,2))},\\
	a_3 &= x_{11} + x_{15}
	= \octavec{disp(x_11)}+ \octavec{disp(x_15)} = \octavec{disp(mod(x_11+x_15,2))}.
\end{align*}
The majority vote is in favour of $0$ and therefore $a_3 = 0$.

With, $x_1x_2\ldots x_{15}=1100111100111011$, these give eight values to each $a_3$, namely
\begin{align*}
	a_4 &= x_0 + x_8
	= \octavec{disp(x_0)}+ \octavec{disp(x_8)} = \octavec{disp(mod(x_0+x_8,2))},\\
	a_4 &= x_1 + x_9
	= \octavec{disp(x_1)}+ \octavec{disp(x_9)} = \octavec{disp(mod(x_1+x_9,2))},\\
	a_4 &= x_2 + x_{10}
	= \octavec{disp(x_2)}+ \octavec{disp(x_10)} = \octavec{disp(mod(x_2+x_10,2))},\\
	a_4 &= x_3 + x_{11}
	= \octavec{disp(x_3)}+ \octavec{disp(x_11)} = \octavec{disp(mod(x_3+x_11,2))},\\
	a_4 &= x_4 + x_{12}
	= \octavec{disp(x_4)}+ \octavec{disp(x_12)} = \octavec{disp(mod(x_4+x_12,2))},\\
	a_4 &= x_{5} + x_{13}
	= \octavec{disp(x_5)}+ \octavec{disp(x_13)} = \octavec{disp(mod(x_5+x_13,2))},\\
	a_4 &= x_{6} + x_{14}
	= \octavec{disp(x_6)}+ \octavec{disp(x_14)} = \octavec{disp(mod(x_6+x_14,2))},\\
	a_4 &= x_{7} + x_{15}
	= \octavec{disp(x_7)}+ \octavec{disp(x_15)} = \octavec{disp(mod(x_7+x_15,2))}
\end{align*}
The majority vote is in favour of $1$ and therefore $a_4 = 1$.

Now we have $\bm{a} = a_{0} 0 1 0 1$ where $a_0\in GF(2)$. Therefore, the original message transmitted was either $\bm{a} = 0 0 1 0 1$ or $\bm{a} = 1 0 1 0 1$. To determine which, assuming at most three errors in the received vector, we calculate $\bm{x} =\bm{a}G$ for each of the two possibilities for $\bm{a}$, where $G$ is the generator matrix given previously.  We then compare the two generated vectors of $\bm{x}$ with the received vector and the one that has three or less differences in the coordinate positions is the one transmitted. Thus,
\begin{align*}
	\bm{x}&=
	\begin{pmatrix}[lrrrrr]
	\octavec{disp(a_1)}
	\end{pmatrix}
	\begin{pmatrix}[lrrrrrrrrrrrrrrrr]
     1 &1 &1 &1 &1 &1 &1 &1 &1 &1 &1 &1 &1 &1 &1 &1 \\
     0 &1 &0 &1 &0 &1 &0 &1 &0 &1 &0 &1 &0 &1 &0 &1 \\
     0 &0 &1 &1 &0 &0 &1 &1 &0 &0 &1 &1 &0 &0 &1 &1 \\
     0 &0 &0 &0 &1 &1 &1 &1 &0 &0 &0 &0 &1 &1 &1 &1 \\
     0 &0 &0 &0 &0 &0 &0 &0 &1 &1 &1 &1 &1 &1 &1 &1
    \end{pmatrix}\\
    &= \octavec{disp(xx_1)}.
\end{align*}
Comparing the vector generated above with the received vector:
\begin{align*}
	&\octavec{disp(xx_1)}\\
	&\octavec{disp(x)}
\end{align*}
shows that they differ in more than three positions and as such the choice of $a_0=0$ was incorrect.
\begin{align*}
	\bm{x}&=
	\begin{pmatrix}[lrrrrr]
	\octavec{disp(a_2)}
	\end{pmatrix}
	\begin{pmatrix}[lrrrrrrrrrrrrrrrr]
     1 &1 &1 &1 &1 &1 &1 &1 &1 &1 &1 &1 &1 &1 &1 &1 \\
     0 &1 &0 &1 &0 &1 &0 &1 &0 &1 &0 &1 &0 &1 &0 &1 \\
     0 &0 &1 &1 &0 &0 &1 &1 &0 &0 &1 &1 &0 &0 &1 &1 \\
     0 &0 &0 &0 &1 &1 &1 &1 &0 &0 &0 &0 &1 &1 &1 &1 \\
     0 &0 &0 &0 &0 &0 &0 &0 &1 &1 &1 &1 &1 &1 &1 &1
    \end{pmatrix}\\
    &= \octavec{disp(xx_2)}.
\end{align*}
Comparing the vector generated above with the received vector:
\begin{align*}
	&\octavec{disp(xx_2)}\\
	&\octavec{disp(x)}
\end{align*}
shows that they differ in three positions and as such the choice of $a_0=1$ was correct. Thus, the original message word was $\bm{a} = \octavec{disp(a(2,:))}$.

\end{enumerate}
